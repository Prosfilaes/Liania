\documentclass{report}
\title{Liania: The Player's Guide}
\author{David Starner}
\begin{document}
\maketitle
\emph{A thousand years ago, at the end of a world-shaking war, the artifacts of
Chaos were buried in an ancient dungeon. Above this dungeon, a city was built,
and in its walls magic was bound to prevent any from leaving the city or the
dungeon below. For all the disturbances of the last millennium, the forces
of Chaos have been relatively quiet; will your explorations keep them from
erupting or accidentally unchain them?}

\chapter{Setting}

% The ancient phrase \emph{panem et circenses} is oft thrown at the inhabitants of the slums, disregarding the fact that
% they would starve without the bread dole, and the amount of work done to supress the political activity of the slums.

\chapter{Ancestries}

There are many ancestries open to players: dwarves, elves, gnomes, goblins,
halflings, orcs and humans (from \emph{Core Rulebook}); leshy and lizard\-folk (from
\emph{Lost Omens Character Guide}); shoony (from AP \#153 / https://2e.aonprd.com);
steel\-heart (from \emph{Pony\-finder: Second Edition Conversion Guide}); various forms
of beast\-folk (from \emph{Pony\-finder: Untamed Lands}); nedjes sphinxes (from \emph{Pony\-finder:
Depths of Everglow}); felsines (from \emph{Felsine}); yroo\-metjis (from
\emph{Files for Everybody: Yroometjis}); mogogols (from
\emph{Amazing Ancestries: The Mogogol}); oya\-poks (from
\emph{Cultures of Celmae: Oyapok 2e}); and ento\-bians (from
\emph{Amazing Ancestries: The Entobian}). (Ratfolk and high fey are works in
progress.)

Humans, halflings, gnomes, goblins and ratfolk are common in Liania.
Orcs, elves, dwarves, leshy, lizardfolk, shoony, steelheart, nedjes sphinxes, and mogogols are
uncommon; everyone has seen them, but few know one personally. The rest are
rare to the point of being unheard of; if no player plays one, it's possible
they will be non-existent in town.

The elves, dwarves and orcs were around before the young races, or so they claim.
It is true that they do not have gods and cannot gain power from the gods; there
are no clerics or champions of these races. In contrast, the young races cannot
cast primal or divine spells without connection to a god. Many elves and
dwarves claim that the bringing of the gods also brought Chaos and the destruction
that came with.

\section {Common races}

\subsection{Humans}

Humans come in a wide variety of forms, in a wide variety of cultures. The
groups that came to Liania a thousand years ago have formed a genetically and
culturally homogenous group, but newer additions are constantly arriving. They
speak Common, though a dozen historical languages are still taught. They both
have many of the lordships and make up the largest part of the slums.

For half-elves, see elves, and likewise for half-orcs. There is no crossbreeding
of ancestries in the world of Liania.

\section {Uncommon races}

\subsection{Elves}

Half-elves are not descended from humans. Upon the coming of the young races,
elves of the original lifespan and powers started becoming rarer; elvish
children started more similar to humans and dying younger. Elves do throw around
the phrase ``half elf'', in Elvish, as an insult; there are technical terms in Elvish for it,
but it is not discussed in polite company. They are just elves. This is not
discussed in mixed company, so while other races can tell the difference
visually, they don't have a clear understand of what's going on.

Elves speak Common Elvish (or just Elvish), as well as either Quenya or Sindarin.
While closely related to each other and more distantly to Common Elvish, the
language signifies a deep split in the Elvish community, and relationships
across the barrier are rare. Newcomers to Liania, particularly those of rare
heritages like Arctic Elves, may speak a different Elvish 
tongue in addition to Common Elvish; they will be pressured to pick a side.

\subsection{Orcs}

There are no rules for playing full orcs yet.

Half-orcs, like half-elves, are not descended from humans. The Orcish term for
them would more literally be translated "adult cub"; as orcs go, it is less
precisely used, being applied to full orcs that are less powerful, and not to
any "half-orc" who could pound the speaker into the ground for such an insult.

\subsection{Dwarves}



\subsection{Steelheart}

\emph{From \emph{Pony\-finder: Second Edition Conversion Guide}; see the GM
for more details.}

The steelheart are a race of mechanical intelligent ponies, numbering about 450.
They were created in town 300 years ago, and the entire race lives here. They
do not reproduce, instead continuously repairing older individuals. Out of the
original 600, 100 have gone missing, and 50 are non-functional and awaiting
repairs. Steelheart with sufficient damage often lack their original memories
after being repaired, but are still considered the same person. (In theory,
this functions as a free raise dead for steelheart; in practice, it will
take equivalent amount of funds to put a PC to the top of the queue. Some
of the non-functional steelheart have been waiting repairs for decades.)

Their creator ascended shortly after their creation, and Fekoro is now a god
worshipped virtually exclusively by the steelheart. Nobody has managed to create
more living constructs like the steelheart, and the steelheart have not been
willing to share any information they have with anyone else.

\section{Leshy}

Cf. Chili Leshy (from \emph{Chili Leshy})

\end{document}
