\documentclass{report}
\title{Liania: The Player's Guide}
\author{David Starner}
\begin{document}
\maketitle
\emph{A thousand years ago, at the end of a world-shaking war, the artifacts of
Chaos were buried in an ancient dungeon. Above this dungeon, a city was built,
and in its walls magic was bound to prevent any from leaving the city or the
dungeon below. For all the disturbances of the last millennium, the forces
of Chaos have been relatively quiet; will your explorations keep them from
erupting or accidentally unchain them?}

\chapter{Setting}

% The ancient phrase \emph{panem et circenses} is oft thrown at the inhabitants of the slums, disregarding the fact that
% they would starve without the bread dole, and the amount of work done to supress the political activity of the slums.

\chapter{Ancestries}

There are many ancestries open to players: dwarves, elves, gnomes, goblins,
halflings, orcs and humans (from \emph{Core Rulebook}); leshy and lizard\-folk (from
\emph{Lost Omens Character Guide}); catfolk, kobolds, orcs, ratfolk and tengu (from
\emph{Pathfinder Advanced Player's Guide}; shoony (from AP \#153);
steel\-heart (from \emph{Pony\-finder: Second Edition Con\-version Guide}); neddies (as nedjes sphinxes, from \emph{Pony\-finder:
Depths of Everglow}); felsines (from \emph{Eldritch Ancestries: Felsine});
(High fey are works in progress.)

Humans, halflings, kobolds, goblins and rat\-folk are most common in Liania.
Dwarves, elves, felsines, leshy, lizardfolk, neddies, shoony, and steelheart are
uncommon; everyone has seen them, but few know one personally. (The list of
rare races have been removed, but they are still available upon discussion.)

The elves, dwarves, kobolds and orcs were around before the young races, or so they claim.
It is true that they do not have gods and cannot gain power from the gods; there
are no clerics or champions of these races. In contrast, the young races cannot
cast primal or divine spells without connection to a god. Many of the old races
blame the bringing of the gods for bringing Chaos and the destruction that came with.

High fey come directly from the fey lands. They cannot gain power from the gods, but like the young
races, they need to worship to cast primal or divine spells, in their case
the Eldest (from \emph{Lost Omens: Gods and Magic}).

All races can interact with the Chaos lords, and this is an exception to the
rule that old races can't become clerics or champions. This is not a PC option.

Aasimars, tieflings, duskwalkers and dhampirs are not available as heritages.
Hobgoblins and gnomes are not available as ancestries.

All races in the city speak Common, even if not otherwise noted.

\section {Common races}

\subsection{Humans}

Humans come in a wide variety of forms, in a wide variety of cultures. The
groups that came to Liania a thousand years ago have formed a genetically and
culturally homogenous group, but newer additions are constantly arriving. They
speak Common, though a dozen historical languages are still taught. They both
have the majority of the lordships and make up the largest part of the slums.

For half-elves, see elves, and likewise for half-orcs. There is no crossbreeding
of ancestries in the world of Liania.

\subsection{Halflings}

Halfings were nearly wiped out in the wars against Chaos, and it is not known
if there are any living outside the city. They are a small, very tight community
with their own walled subcity. They are somewhat notorious; those that live
outside the halfling city and thus interact with others more often tend to be
exiles and troublemakers. The city did compel the halfling subcity to report
murderers and other unsavory characters, after an incident with a serial killer
one hundred and fifty years ago, but it is still true that many who commit crimes
are quietly exiled instead of the information being passed to the larger city.

Halflings speak Halfling and Pretorius, a religious language.
They usually worship Decimus, a LN ascended halfling from long before Liania, though
there is a heresy of Julia, a NG ascended halfling alleged to be his daughter,
and the seperate religion of Terious, a TN ascended halfling that claims to represent the world itself.

\subsection{Kobolds}

Kobolds make up a lot of the slums and poverty level labor, as well as much of
the organized crime and revolutionary groups. They tend to make much of their
being an old race, and claim to be older than the other old races. Elves and
dwarves tend to ignore this, but it riles up certain orcs to the point of
violence \ldots\ which possibly explains why many kobolds only bring this up
in the presence of orcs.

Kobolds speak Kobold. They assert this to be Draconic; given that there are no
full dragons in the city, and Draconic is not historically written, this is
doubted but undisprovable.

\subsection{Goblins}

Many people ask themselves why the city still has to deal with goblins? There
have been constant attempts at dealing with them, which twice have extended to
open genocide. Currently the city makes sure that some goblin gets punished for
any acts of arson, even if good people keep bleating about how inappropriate
it is to just round up goblins without any real evidence of their personal
guilt.

Goblins yet live. Some of them take good jobs, many of them live on the grain
dole and amuse themselves in relatively undestructive ways and some of them
go on wild rampages. They also tend to hide below ground, and frequently a
group of goblins that have lived in the dungeons a few generations will return
to the city. A few goblins are deep servitors of Chaos lords, a fact
fortunately hiden by their less disciplined kin.

In addition to the Chaos lords, goblins worship a plethora of goblin gods.
About a dozen actually correspond to ascended goblins, but all of these gods
have enough alchemists, non-divine sorcerers, bards and/or stage magicians
attributing power to them to make it unclear to anyone which 12 are actual
gods.

Goblins speak a host of ever-mutating languages, which ratfolk scholars have
labeled current dialects Goblin-1 through Goblin-6. If two goblins speaking
different tongues meet, they will depend on the principle of speaking louder and
slower, and hopefully cognates will get enough of the message through. Goblins
also know Common and will fall back on it in emergencies, but are convinced
they all speak the same language, the others just speaking it poorly.

\subsection{Ratfolk}

The ratfolk are newcomers to the city, having only been here about 250 years.
They notably lack in political power due to this, but have a great deal of social
power. They trade everything, and they are entrenched in the magical and academic
hierachies. Despite this, they get a lot of distrust, and are considered outsiders.
Ratfolk are often the employers of goblins, given that they feel commonality for
being outsiders, though the generally low wages goblins get paid are a cause for
backlash in certain sectors of the goblin community.

Ratfolk speak Ratfolk. They worship a quadrumvirate of gods, Tskhovreba (NG
god of life and growing things), Tsodna (LN god of education and knowledge),
Misnoba (N god of power and magic), and Tavisupleba (CN god of anarchy and
freedom).

\section {Uncommon races}

\subsection{Dwarves}

Dwarves are distinctly a different type of being than the other humanoids. They
are creatures of stone-infused flesh. They will sometimes let themselves be categorized
as male or female, but aren't really either. A dozen dwarves will engage in a group
ritual to animate a large lump of stone into a large-sized baby dwarf, which will
wear down to a normal sized dwarf over time. While growing up, the young dwarf
will be the responsibility of a single ``parent'', who may or may not be one of
the original animators, for five to ten years before being passed to another
``parent''.

Dwarves have skin the color of the stone used to make them; granite is common
in Liania, but the dwarves will frequently import stone for forming children.
As they grow up, they become more dwarven in form and start to develop humanoid
hair, teeth, eyes and nails. Hair in particular grows in the color, style and
length chosen by the dwarf, though they may have to shed to start growing a new
style.

Dwarves speak Dwarvish. They have a very distinct religion that offers minor
offerings of plant to water, air and fire, with prayers being centered around
proper balance.

\subsection{Elves}

Half-elves are not descended from humans. Upon the coming of the young races,
elves of the original lifespan and powers started becoming rarer; elvish
children started more similar to humans and dying younger. Elves do throw around
the phrase ``half elf'', in Elvish, as an insult; there are technical terms in Elvish for it,
but for most polite purposes, they are just elves. This is not
discussed in mixed company, so while other races can tell the difference
visually, they don't have a clear understand of what's going on.

Elves speak Common Elvish (or just Elvish), as well as either Quenya or Sindarin.
While closely related to each other and more distantly to Common Elvish, the
language signifies a deep split in the Elvish community, and relationships
across the barrier are rare. Newcomers to Liania, particularly those of rare
heritages like Arctic Elves, may speak a different Elvish
tongue in addition to Common Elvish; they will be pressured to pick a side.

\subsection{Felsines}

\emph{From \emph{Eldritch Ancestries: Felsine}); see the GM for more details}

Felsines are a small, cute relative of catfolk. While not part of the original
races of Liania, the first batch showed up less than a hundred years after the
founding, with more groups following at regular intervals. They are frequently
underestimated, and felsines sometimes lean into this, but people who get to
know them know of their determination, patience and occasional viciousness.

Felsines speak Felsine and a language of their particular tribe, in addition to
Common. They worship a plethora of gods, both from the seven ascended Felsines
and those of other races.

\subsection{Orcs}

Half-orcs, like half-elves, are not descended from humans. The Orcish term for
them would more literally be translated ``adult cub''; as orcs do, it is less
precisely used than Elvish ``half-elf'', being applied to full orcs that are less powerful, and not to
any ``half-orc'' who could pound the speaker into the ground for such an insult.

Half-orcs and orcs are subject to quite a bit of suspsicion in Liania, but their
status as an old race does provide them a bit of protection from the extremes
of the treatment of the goblins. When they do have jobs, they tend to be ones
where they can crack heads; the general reputation of orcs as unintelligent
make few, even other orcs, willing to follow them.

Orcs speak Common; there was an Orcish language, but if there number any speakers
of it remaining among the orcs, they number quite few.

\subsection{Shoony}

https://2e.aonprd.com/Ancestries.aspx?ID=16


\subsection{Steelheart}

\emph{From \emph{Pony\-finder: Second Edition Conversion Guide}; see the GM
for more details.}

The steelheart are a race of mechanical intelligent ponies, numbering about 450.
They were created in town 300 years ago, and the entire race lives here. They
do not reproduce, instead continuously repairing older individuals. Out of the
original 600, 100 have gone missing, and 50 are non-functional and awaiting
repairs. Steelheart with sufficient damage often lack their original memories
after being repaired, but are still considered the same person. (In theory,
this functions as a free raise dead for steelheart; in practice, it will
cost as much as a normal Raise Dead to move a PC to the top of the queue. Some
of the non-functional steelheart have been waiting repairs for decades.)

Their creator ascended shortly after their creation, and Fekoro is now a god
worshipped virtually exclusively by the steelheart. Nobody has managed to create
more living constructs like the steelheart, and the steelheart have not been
willing to share any information they have with anyone else.

\section{Leshy}

Cf. Chili Leshy (from \emph{Chili Leshy})

\end{document}
